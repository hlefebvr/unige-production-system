\section{Continious review}
\subsection{Deterministic case}

\par Let's define $\lambda$ the rate of "emptying" the storage as $\lambda=\frac{items}{t}$

\par [Schema]

\par Costs associated to inventory management are of different types, we denote :  
\begin{itemize}
    \item $K$ : fixed cost for ordering a product (€ / order)
    \item $h$ : inventory cost (€ / t.items)
    \item $c$ : unitary cost (€ / item)
\end{itemize}
And let's call $Q$ the amount of goods we order per delivery (note that $Q=\lambda T$). The goal is to find what is the optimal amount of goods to order knowing the different prices and the "delivery" time given by the supplier ($\tau$).

We want to optimize our inventory management with respect to its cost for a given period of time : 

\[
    \begin{split}
    J &= \left( \underbrace{K}_{\textrm{fixed cost per order}} + \underbrace{cQ}_{\textrm{price to buy }Q\textrm{ items}} + \underbrace{\int_0^ThI(t)dt}_{\textrm{storage cost}} \right) \times \frac{1}{T} \\
    &= \left( K + cQ + h\frac{QT}{2} \right) \times \frac{\lambda}{Q} \\
    &= \frac{K\lambda}{Q} + c\lambda + \frac{hQ}2
    \end{split}
\]

Let's take the derivative with respect to $Q$ :
\[
    \frac{\partial J}{\partial Q} = -\frac{K\lambda}{Q^2} + \frac{h}{2} \Rightarrow Q^* = \sqrt{\frac{2K\lambda}{h}}
\]
This point is called the \begin{textbf}{Economic Order Quantity}\end{textbf} (EOQ). 

Knowing the delivery time of our supplier $\tau$, we can compute the exact reorder point at which one should call the supplier to order $Q^*$ goods : 
\[ R = \lambda\tau \]

\subsection{Stochastic case}

\[
    \begin{split}
    \tau = \bar\tau + \sigma_\tau x & \textrm{ : stochastic delivery time}\\
    d = \bar d + \sigma_d x & \textrm{ : stochastic demand }\\
    x\sim\mathcal{N}(0,1)
    \end{split}
\]

The reorder point considered is now given by $\bar\tau\bar d + s.stock$ where the safety stock is some amount of goods reserved in the warehouse in order to lower the probability of being under delivered.

The associated "average" cost  is now given by : 
\[
    J = \underbrace{\frac{K\bar d}Q + c\bar d + \frac{hQ}2}_{f(Q)} + \underbrace{h\times s.stock}_{f(s.stock)}
\]
Since this function is perfectly splittable into two functions of one variable each, we can seperately minimize them with respect to the cost.
Minimizing the first part is similar to what we did in the deterministic case : 
\[ Q^* = \sqrt{\frac{2K\bar d}h} \]

Two strategies can be used to choose a "good" safety stock level : 
\begin{enumerate}
    \item Choosing the re-order point $R$ so that the probability of having less demand than we currentely have in store is greater to a certain number (risk).
    \[
        P_\tau (d_\tau\le R)\ge\alpha
    \]
    where $\alpha$ is called the "fill-rate" and with $d_\tau=\bar d_\tau+\sigma_{d_\tau}x$ where
    \[
        \begin{split}
        \bar d_\tau &= \bar d \bar\tau \\
        \sigma_{d_\tau} &= \sqrt{ \sigma_d^2\bar\tau + \sigma^2\bar d^2 }
        \end{split}
    \]
\end{enumerate}